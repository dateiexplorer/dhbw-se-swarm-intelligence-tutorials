%!TEX root = ../thesis.tex

\section{Metapher}

Der Artificial Bee Colony (ABC) Algorithmus ist vom Verhalten von Honigbienen
inspiriert, die Nektar von Blumen sammeln und als Futter in den Bienenstock
bringen. Dabei werden zunächst Kundschafterbienen (\emph{scout bees}, im Folgenden
\emph{Scouts} genannt) ausgesendet, die nach Regionen suchen, in denen Nektar
zu finden ist. Diese kehren schließlich zurück in den Bienenstock und
informieren andere Bienen über die Fundstelle und Qualität (\emph{fitness})
der Futterquelle.
Daraufhin kehren die Scouts zur Futterstelle zurück. Einige andere Bienen
folgen ihnen. Ein kleiner Teil der Scouts sucht daraufhin nach neuen
Futterquellen, während andere Bienen weiterhin die Qualität einer
Futterquelle bei jeder Rückkehr in den Bienenstock über einen sog. Bienentanz
kommunizieren.

Dieses Verhalten sorgt für eine effiziente Nahrungssuche bei Honigbienen und
wird für die iterative Suche nach einer Lösung von Optimierungsproblemen im
ABC immitiert. Gesucht ist dabei die Biene, die die Futterquelle mit der besten
Fitness findet.

\section{Strategie}

Um ein globales Optimum zu finden, wird eine Honigbienenkolonie imitiert, die
drei Arten von Bienen beinhaltet, die sog. \emph{Employed Bees} (beschäftigten
Bienen), die \emph{Onlookers} (Zuschauerbienen) und die \emph{Scouts}
(Kundschafterbienen). Eine Hälfte der Kolonie besteht aus Employed Bees, die
andere Hälfte beinhaltet die Onlookers.

Futterquellen repräsentieren Lösungen eines Problems, das mit dem ABC
gelöst werden soll. Je höher die Qualität bzw. Fitness einer Lösung ist, desto
größer ist die Menge an Nektar. Der Algorithmus konvergiert, indem die Bienen
iterativ das globale Optimum finden, da sie mit der Zeit die Futterquellen mit
der besten Fitness ansteuern. Diese Futterquelle stellt die beste Lösung des
Optimierungsproblems dar.

Scouts suchen zufällig nach einer neuen Futterquelle und ersetzen damit
Futterquellen, also Lösungen, die sich nicht bewährt haben, deshalb von den
Bienen verlassen wurden und damit nicht das globale Optimum darstellen.

Employed Bees suchen nach Futterquellen in ihrer direkten Nachbarschaft, die
mehr Nektar haben, als die aktuelle Futterquelle, auf der sie sich befinden.
Die Suche ist zwar zufällig, bezieht allerdings das globale Wissen des
Schwarms über andere Futterquellen mit ein. Diese Art von Bienen sucht nach
neuen Lösungen und teilt ihre Ergebnisse den Onlookers mit.

Onlookers wiederum beziehen die Positionen der Futterquellen von den
Employed Bees. Diese Bienen wählen stetig bessere Lösungen aus und suchen in
ihrer Nachbarschaft nach weiteren Lösungen.

Der ABC besteht aus drei Phasen. In der ersten Phase des Alogrithmus werden
verschiedene Positionen im Suchraum zufällig als Futterquellen ausgesucht und
deren Fitness ermittelt.
In der zweiten Phase schwärmen die Employed Bees aus und suchen nach neuen
Futterquellen in direkter Nachbarschaft zu den initialen Futterquellen.
Wenn die Onlookers ausschwärmen, präferieren sie die Positionen, die
am meisten Nektar produzieren. Diese Information wird ihnen durch die
Employed Bees mitgeteilt. Wenn die Menge an Nektar ansteigt, steigt auch
die Wahrscheinlichkeit, dass Onlookers diese Futterquelle auswählen. Diese
Auswahl erfolgt in der dritten Phase des Alogorithmus.
Wird die Zielposition von einer Biene erreicht, so wird nach einer neuen
Futterquelle in der Umgebung in der direkten Nachbarschaft gesucht.
Wird eine Futterquelle verlassen, wird diese von den Scouts zufällig durch
eine neue ersetzt.

So wird der gesamte Suchraum iterativ nach dem Optimum (Futterquelle mit der
besten Fitness) abgesucht.

\section{Prozedur}

Zunächst werden zufällig Positionen im Suchraum ausgewählt. Die Anzahl der
Positionen ist dabei von der Population abhängig. Jede Position (jede
Futterquelle) ist ein $n$-dimensionaler Lösungsvektor des
Optimierungsproblems, wobei $n$ die Anzahl der Optimierungsparameter
darstellt.

Nach der Initialisierung werden die einzelnen Positionen zyklisch von
Emplyed Bees, Onlookers und Scouts besucht. Employed Bees aktualisieren
dabei ihre aktuelle beste Position in Abhängigkeit der lokalen Information
über die Fitness ihrer Futterquelle und testet die Menge an Nektar
(Fitness) von neuen Quellen. Diese neuen Futterquellen sind mögliche neue
Lösungen des Optimierungsproblems.
Wenn die Nektarmenge der neuen Lösung höher ist als die momentane, wird die
beste lokale Position einer Employed Bee aktualisiert. Ist die Nektarmenge
kleiner als die der aktuellen Position, so merkt sich die Biene diese als
beste lokale Position.

Nachdem alle Bienen ihren Suchvorgang abgeschlossen haben, teilen sie die
neu gewonnen Informationen über die Menge an Nektar und die dazugehörigen
Positionen den Onlookers mit. Die Onlookers berücksichtigen bei der Auswahl der
Position, die sie anfliegen, die Menge an Nektarvorkommen aller gefundenen
Futterquellen. Die Wahrscheinlichkeit, mit der eine Biene eine Futterquelle
anfliegt, berechnet sich nach der folgenden Formel:
\begin{equation}
    p_i = \frac{\mathrm{fit}_i}{\sum_{j=1}^n \mathrm{fit}_j}
\end{equation}
wobei $\mathrm{fit}_i$ die Fitness bzw. die Menge an Nektar der Futterquelle
einer Lösung $i$ und $n$ die Anzahl an Futterquellen ist. Diese Anzahl ist
gleich der Anzahl an Employed Bees.

Um aus dem Gedächtnis einen neuen Kanditat für die Futterquelle zu erzeugen,
kann die folgende Formel angewendet werden:
\begin{equation}
    v_{ij} = x_{ij} + \Phi_{ij}(x_{ij} - x_{kj})
\end{equation}
wobei $k$ ein zufälliger Index der möglichen Lösungen und $j$ ein zufälliger
Index des Vektors an Optimierungsparametern ist. $\Phi$ ist eine zufällige
Zahl zwischen $[-1;1]$.

\section{Pseudocode}

Zu Beginn wird die Population der Kolonie initialisiert. Dabei werden genauso
viele Employed Bees erzeugt, wie es Lösungen für das Problem gibt. Diese
Größe wird als \code{problemSize} angegeben.
Anschließend wird iterativ das globale Optimum ermittelt. Eine
\code{while}-Schleife läuft so lange, bis das globale Optimum gefunden wurde
oder eine andere Abbruchbedingung erfüllt ist, beispielsweise die maximale
Anzahl an Zyklen, die bei Programmstart festgelegt werden, erreicht wurde.

Bei jeder Iteration bewerten zunächst die Employed Bees die Fitness ihrer
aktuellen Position. Anschließend ermitteln die Bienen die Position mit der
besten Fitness ihrer Nachbarn und schwirren dorthin aus.

Die Informationen über die neuen Futterquellen werden dem kollektiven
Gedächtnis hinzugefügt. Daraufhin suchen die Onlookers eine beste
Futterquelle, basierend auf der Fitness dieser, aus.

Ist nach diesem Schritt keine Biene mehr an der aktuellen Position, so wird
diese Futterquelle aufgelöst und die Employed Bee wird zum Scout und sucht
eine zufällige andere Position im Suchraum.

Wenn der Algorithmus terminiert, liefert er als Ergebnis die Biene mit dem
besten Ergebnis zurück, dem globalen Optimum.

\section{Exploration and Exploitation}

Die Exploration~--~also das Erkunden neuen Suchraums~--~wird beim ABC durch die
zufällige Auswahl der Futterquellen durch die Scouts sichergestellt. Dadurch,
dass Futterquellen nach jeder Iteration von den Employed Bees nach ihrer
Fitness bewertet werden und Onlookers dementsprechend Positionen auswählen,
wird sichergestellt, dass bekannter Suchraum nach jeder Iteration weiterhin
genauer untersucht wird (Exploitation). Gleichzeitig werden Positionen, die
keinesfalls das globale Optimum bieten, schnell aussortiert und durch neue
Positionen im Suchraum ersetzt.

Damit kann der Suchraum effektiv exploriert werden (sichergestellt durch
Scouts und Employed Bees, die ihre noch unbekannte Nachbarschaft mitbetrachtet)
und bereits bekannter Suchraum exploitiert werden (durch Employed Bees, die
ihre eigene Futterquelle bewerten und Onlooker Bees, die nur Positionen
anfliegen, die ihnen von den Employed Bees mitgeteilt werden).

\section{Entscheidungsregeln für Schwarmverhalten}

Employed Bees wählen Futterquellen zufällig im bekannten Suchraum aus. Diese
Informationen werden anderen Employed Bees und den Onlookers durch das
kollektive Gedächtnis zugänglich gemacht.
Mithilfe dieser Informationen suchen die Onlookers das globale Optimum, indem
sie die Futterquellen auf ein fitnessabhängiges Rouletterad legen und
wahrscheinlichkeitsbasiert auswählen.
Employed Bees suchen nach dem lokalen Optimum, indem sie sich
Nachbarfutterquellen ansehen und die mit der besten Fitness auswählen.
Scouts suchen zufällig neue Futterquellen im unbekannten Suchraum.

\section{Parameterabhängigkeiten}

Die \emph{Anzahl der Zyklen} gibt die maximale Anzahl an Durchläufen an, bis
der Algorithmus terminiert wird. Wird dieser Wert zu klein gewählt, besteht
die Gefahr, dass das tatsächliche globale Optimum noch nicht gefunden wurde,
ist der Wert hingegen zu groß, werden unnötige Zyklen durchlaufen, in denen
sich das globale Optimum nur noch geringfügig oder nicht mehr verändert.

Die \emph{Anzahl der Futterquellen} entspricht der initialen Anzahl an
Employed Bees (und damit auch der Onlookers), die jeweils die Hälfte der
Gesamtpopulation ausmachen. Eine größere Anzahl an Futterquellen erhöht die
Komplexität pro Zyklus und benötigt damit mehr Rechenaufwand und Zeit.
Gleichzeitig kann ein genaueres Ergebnis erzielt werden. Hier ist eine
Abwägung zwischen Genauigkeit und Komplexität individuell für das
Optimierungsproblem zu treffen.

\section{Zusammenspiel \emph{Employed Bee}, \emph{Onlooker Bee}, und
    \emph{Scout Bee}}

Mit einer Konfiguration von einer Anzahl an Futterquellen von 100 und einem
Zyklenlimit von 25 wird zunächst eine Population von 100 Employed Bees und
100 Onlookers initialisiert, da für jede Futterquelle eine Employed Bee
benötigt wird und diese die Hälfte der gesamten Population stellen.
Employed Bees ermitteln die Fitness ihrer Futterquellen und teilen sie den
Onlookers durch das kollektive Gedächtnis (Bienentanz) mit. Daraufhin
schwirren die Onlookers wahrhscheinlichkeitbasiert zu den Futterquellen mit
der besten Fitness. Wird eine Futterquelle nicht angeflogen, so wird die
für diese Quelle zuständige Employed Bee zum Scout und sucht eine neue
Futterquelle im unbekannten Suchraum.
Dieser Vorgang wird zyklisch wiederholt. Nach 25 Zyklen konvergiert der
Algorithmus bereits. Er wird terminiert und das bisherige globale Optimum
wird zurückgegeben.