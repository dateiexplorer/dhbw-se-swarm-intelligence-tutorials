%!TEX root = ../thesis.tex

\section{Metapher}

Der Artificial Bee Colony (ABC) Algorithmus ist vom Verhalten von Honigbienen
inspiriert, die Nektar von Blumen sammeln und als Futter in den Bienenstock
bringen. Dabei werden zunächst Suchbienen (\emph{scout bees}, im Folgenden
\emph{Scouts} genannt) ausgesendet, die nach Regionen suchen, in denen Nektar
zu finden ist. Diese kehren schließlich zurück in den Bienenstock und
informieren andere Bienen über die Fundstelle und Qualität (\emph{fitness})
der Futterquelle.
Daraufhin kehren die Scouts zur Futterstelle zurück. Einige andere Bienen
folgen ihnen. Ein kleiner Teil der Scouts sucht daraufhin nach neuen
Futterquellen, während andere Bienen weiterhin die Qualität einer
Futterquelle bei der Rückkehr in den Bienenstock kommunizieren.

\section{Strategie}

Um ein globales Optimum zu finden, wird eine Honigbienenkolonie imitiert, die
drei Arten von Bienen beinhaltet, die sog. \emph{Employed Bees}, die
\emph{Onlookers} und die \emph{Scouts}. Eine Hälfte der Kolonie besteht aus
\emph{Employed Bees}, die andere Hälfte beinhaltet die \emph{Onlookers}.
Die Futterquellen repräsentieren Lösungen eines Problems, das mit dem ABC
gelöst werden soll. Je höher die Qualität bzw. Fitness einer Lösung ist, desto
mehr Necktar produziert sie. Der Algorithmus konvergiert, indem die Bienen
iterativ näher an das globale Optimum herankommen, da sie die Futterquellen mit
der besten Fitness ansteuern.

Scouts suchen zufällig nach einer neuen Futterquelle und ersetzen damit
Futterquellen, also Lösungen, die sich nicht bewährt haben und damit nicht das
globale Optimum darstellen.

Employed Bees suchen nach Futterquellen in ihrer direkten Nachbarschat, die
mehr Nektar haben, als die aktuelle Futterquelle, auf der sie sich befinden.
Die Suche ist zwar zufällig, bezieht allerdings das globale Wissen des
Schwarms über andere Futterquellen mit ein. Diese Art von Bienen sucht nach
neuen Lösungen und teilt ihre Ergebnisse den Onlookers mit.

Onlookers wiederum beziehen die Positionen der Futterquellen von den
Employed Bees. Diese Bienen wählen stetig bessere Lösungen aus und suchen in
ihrer Nachbarschaft nach weiteren Lösungen.

% TODO: Hier noch mehr zur Strategie einfügen

\section{Prozedur}

\section{Pseudocode}

\section{Exploration and Exploitation}

\section{Entscheidungsregeln für Schwarmverhalten}

\section{Parameterabhängigkeiten}

\section{Zusammenspiel \emph{Employed Bee}, \emph{Onlooker Bee}, und
    \emph{Scout Bee}}