%!TEX root = ../../thesis.tex

\chapter{Basics}

In diesem Kapitel werden nun die grundlegenden \LaTeX{}-Funktionen vorgestellt.
Es wird empfohlen, einzelne Textbausteine in den Quelldateien nachzulesen, da
dort einige weitere Informationen zu finden sind.

\section{Kompilieren einer \LaTeX{}-Datei}

Zunächst ist wichtig zu verstehen, dass es sich bei den \emph{.tex} Dateien um
ganz normale Textdateien handelt, die mit einem herkömlichen Texteditor
einsehen und bearbeiten lassen. Damit ist \LaTeX{} unabhängig von irgendeiner
zusätzlichen Software. Dateien können auch komplett im Terminal mit Texteditoren
wie \emph{nano} oder \emph{vim} unter Linux entwickelt werden.
Diese Textdateien werden anschließend mit einem Programm kompiliert, um daraus
eine PDF-Datei zu erzeugen.

\section{Texte schreiben}

In einer TEX-Datei kann schließlich ganz normaler Text geschrieben werden. Dabei
werden Zeichenumbrüche ignoriert. Das schöne ist, dass \LaTeX{} sich um das
Layout und das Setzen des Textes kümmert. Das heißt, es ist egal, wie unsere
Quelldatei letztlich formatiert ist, am Ende erhält man immer ein Ergebnis, das
nach typographischen Regeln gut aussehen wird.

Mit verschiedenen Commands können Texte auch angepasst werden, dabei wird
mit den Commands immer eine semantische Bedeutung gegeben. Das Command
\code{\textbackslash{}emph} weist \LaTeX{} beispielsweise dazu an, einen Text
hervorzuheben. Standardmäßig wird das durch das kursive Setzen des
entsprechenden Textes bewerkstelligt.

Mit einer leeren Zeile wird \LaTeX{} mitgeteile, dass nun ein neuer Absatz
erfolgen soll. In dieser Vorlage ist standardmäßig eingestellt, dass ein neuer
Absatz durch einen kleinen Zwischenraum abgetrennt wird. In der
Standardeinstellung eines normalen \LaTeX{}-Dokuments, das nicht mit dieser
Vorlage erstellt wird, wird der Absatz eingerückt (wie man es von Büchern her
kennt). In wissenschaftlichen Arbeiten ist dieses Verhalten aber eher unüblich.
Letztlich ist es aber Geschmackssache und jeder kann die Vorlage nach seinen
Wünschen anpassen.

\section{Aufzählungen}

In \LaTeX{} können auch verschiedene Auflistungen gemacht werden. Diese
Aufzählungen können nummeriert oder unnummeriert sein.

Ungeordnete Liste:
\begin{itemize}
    \item Item 1
    \item Item 2
    \item Item 3
\end{itemize}

Geordnete Liste:
\begin{enumerate}
    \item Item 1
    \begin{enumerate}
        \item Item 1.1
        \item Item 1.2
        \item Item 1.3
    \end{enumerate}
    \item Item 2
    \item Item 3
\end{enumerate}

Außerdem gibt es die Möglichkeit, eine Aufzählung von Worten und einer
Beschreibung zu generieren.

\begin{description}
    \item[Item 1] \hfill \\
        Beschreibung des ersten Items in der Liste. Diese Beschreibung kann auch
        über mehrere Zeilen gehen.
    \item[Item 2] \hfill \\
        Beschreibung des zweiten Items...
    \item[Item 3] \hfill \\
        Beschreibung des dritten Items...
\end{description}