%!TEX root = ../thesis.tex

\section{Metapher}

Der Ant-Colony-Optimization-Algorithmus (ACO) ist von dem sozialen Verhalten
von Insekten und insbesondere von Ameisen inspiriert. Dabei ist ein
Individuum einer Kolonie wenig komplex, durch das Zusammenspiel aller
Individuen sind Ameisen allerdings in der Lage, gemeinsam komplexe Aufgaben
zu bewältigen.

Der hier vorgestellte Algorithmus lässt sich von der Futtersuche von Ameisen
ableiten. Das Verhalten wird in dem sog. \emph{Double-Bridge-Experiment}
analysiert. Dabei haben Ameisen genau zwei Wege, um von ihrem Nest zu einer
Futterquelle zu gelangen. Einer der beiden Wege ist dabei länger als der
andere. Zu Beginn schwärmen die Ameisen aus und nutzen beide Wege, um zu
der Futterquelle zu gelangen. Nach einiger Zeit wird man allerdings
beobachten, dass sich eine Ameisenstraße über den kürzeren der beiden Wege
bildet. Die Ameisen nutzen also eine Strategie, um den schnellsten Weg von
ihrem Nest zum Futter zu finden.
Diese Strategie ist maßgeblich auf das Schwarmverhalten und ausgesetzte
Pheromonspuren zurückzuführen. Finden Ameisen Futter, so kommunizieren sie
den Weg ihren Artgenossen durch einen Duftstoff über den Weg, den sie
gelaufen sind. Andere Ameisen folgend schließlich dieser Duftspur und
gelangen ebenfalls an das Futter. Kürzer Wege werden so öfter frequentiert
von Ameisen abgelaufen, wodurch die Pheromonspur auf diesem Weg stärker wird,
während die Pheromonspur auf den anderen Wegen mit der Zeit verfliegt.
So nehmen immer mehr Ameisen die kürzeste Strecke und es bilden sich die
typischen Ameisenstraßen.
Damit nutzen Ameisen ihr Schwarmverhalten, um den Weg von ihrem Nest zur
Futterstelle zu optimieren. Eine Eigenschaft, die sich auch für das TSP
zunutze gemacht werden kann.

\section{Strategie}

Die Strategie für diesen Algorithmus liegt darin, dass sowohl das kollektive
Schwarmverhalten durch die Pheromonspur genutzt wird, um bekannte Wege zu
exploitieren und den günstigsten Pfad zu finden, als auch ein individuelles
Verhalten in die Wegfindung einzelner Ameisen mit einfliest, das durch eine
einfache Heuristik simuliert wird.

Ameisen laufen verschiedene Pfade iterativ ab und teilen die
Pheromoninformationen über das kollektive Gedächtnis in einer Pheromonmatrix,
die jeder Ameise zur Verfügung steht. Auf diese Weise lässt sich effektiv
ein Optimum für den gewünschten Suchraum finden.

\section{Prozedur}
\label{sec:prozedur}

Für das kollektive Gedächtnis wird eine Pheromonenmatrix angelegt, die die
Pheromonenwerte aller Kanten eines Graphen beinhalten. Diese Werte fließen
in die Entscheidung für einen Knotenpunkt bei jeder Iteration mit ein. Die
Pheromonenmatrix ist dabei das kollektive Gedächtnis der Ameisenkolonie.
Jede Ameise nutzt außerdem eine Metaheuristik, mit der sie selbstständig einen
nächsten Knotenpunkt auswählt. Beide Faktoren können gewichtet werden, um
ihren jeweiligen Einfluss auf die Entscheidung einer einzelnen Ameise zu
steuern.
Daraus lässt sich ein stochastisches Modell entwickeln, das sich mathematisch
folgendermaßen ausdrücken lässt:
\begin{equation}
    p_{i,j}^k(t) = \frac{\tau_{i,j}^\alpha * \eta_{i,j}^\beta}{
        \sum_{k=1}^c \tau_{i,k}^\alpha * \eta_{i,k}^\beta
    }
\end{equation}

Dabei ist $p_{i,j}$ die Wahrscheinlichkeit, mit der die Kante des Graphen
von Position $i$ nach Position $j$ von einer Ameise gelaufen wird,
$\tau_{i,j}$ ist die Pheromoninformation für diese Kante und wird über das
kollektive Gedächtnis mit allen Ameisen geteilt, $\eta_{i,j}$ ist die
heuristische Information, die beispielsweise durch die Berechnung der
Euklidischen Distanz ziwschen den einzelnen Punkten im Graph berechnet wird;
$\alpha$ ist die Gewichtung, mit der die Pheromoninformationen, also das
kollektive Gedächtnis, auf die Entscheidung der Ameise Einfluss nehmen und
$\beta$ ist entsprechend die Gewichtung für die Einflussnahme der Heuristik,
also dem individuelle Verhalten, auf die Entscheidung.
Um bessere Ergebnisse zu erzielen, werden die Ergebnisse außerdem ins
Verhältnis zu den Ergebnissen aller Ameisen gesetzt. 

Mit dieser grundlegenden Formel lassen sich verschiedene Pfade im
gewünschten Suchraum ablaufen.
Nach jeder Iteration werden die Pheromonenwerte in der Pheromonenmatrix
aktualisiert. Wird die maximale Anzahl an Iterationen erreicht oder das
Optimum ändert sich nicht oder nur geringfügig, ist das Optimum gefunden.
Der Algorithmus liefert den Pfad, der nach der letzten Iteration als
bester Pfad ausgewählt wurde. Das ist derjenige, auf dem die höchste
Pheromonkonzentration zu finden ist.

\section{Pseudocode}

Der Hauptteil des ACO ist eine einfache \code{while}-Schleife, die solange
durchlaufen wird, bist eine Abbruchbedingung erfüllt ist. Diese kann
beispielsweise durch eine maximale Anzahl an Iterationen festgelegt werden
oder durch einen Fitness-Wert, der die Veränderung zum vorherigen Ergebnis
angibt. Fällt dieser Werte unter eine bestimmte Schranke, ist das Ergebnis
nah genug am Optimum und der Algorithmus kann terminiert werden.
Für den Algorithmus ergibt sich folgender Pseudocode:

\begin{lstlisting}
    while (Abbruchbedingung trifft nicht zu) do
        GenerierePfadeFuerAmeisen();
        AktualisierePheromone();
        AktualisiereMomentanesOptimum();
    done
\end{lstlisting}

In jedem Durchlauf läuft jede Ameise mithilfe der in \ref{sec:prozedur}
beschriebenen Formel einen gesamten Pfad für das TSP ab. Anschließend
wird die Pheromonmatrix für den nächsten Durchlauf aktualisiert.
In diesem Schritt wird auch ein \emph{evaporation}-Faktor (Verdampfung)
angewendet, der angibt, wie viel vom ursprünglichen Pheromonenwert pro
Kante für die nächste Iteration behalten wird. Damit werden weniger
frequentierte Kanten schneller im kollektiven Gedächtnis als
nichtzielführend markiert, wodurch der Algorithmus schneller ein globales
Optimum finden kann.
Außerdem lässt sich der Algorithmus auch bei dynamischen Problemen einsetzen.
Würden sich Ausgangsparameter ändern, kann über die \emph{evaporation}
sichergestellt werden, dass ursprüngliche vielversprechende Lösungen, die
durch die Parameteränderung nicht mehr zielführend sind, mit der Zeit wieder
vergessen werden.
Am Ende jedes Durchlaufs wird der beste Pfad ermittelt. Dieser stellt das
momentane gefundene globale Optimum dar.

\section{Exploration and Exploitation}

Die Exploration, also das Erkunden des neuen Suchraums, wird maßgeblich durch
die Gewichtung $\beta$ der Metaheuristik beeinflusst. Die Erkundung geschieht
dadurch, dass Ameisen an jedem Knotenpunkt im Graphen wahrscheinlichkeitsbasiert
auswählen, welche Kante sie entlanggehen. Je höher $\beta$ im Gegensatz zu
$\alpha$ gewählt wird, desto mehr Suchraum wird exploriert, allerdings leidet
darunter die Exploitation.
Diese wiederum wird durch die Gewichtung $\alpha$ des Einflusses des
kollektiven Gedächtnisses bei der Wegfindung über die Pheromonenmatrix
beeinflusst.
Ein ausgeglichenes Verhältnis zwischen $\alpha$ und $\beta$ ist zu wählen, um
die Suche nach einem Optimum möglichst effizient zu gestalten.

\section{Entscheidungsregeln für Schwarmverhalten}

Ameisen im ACO nutzen zum einen das kollektive Gedächtnis, um
wahrscheinlichkeitsbasiert an jedem Knotenpunkt eines Graphen die nächste
Kante auszuwählen ($\tau$), zum anderen eine Metaheuristik ($\eta$), um ein
individuelles Verhalten einer Ameise zu simulieren. Wie stark diese beiden
Faktoren einen Einfluss auf die Wegfindung einer Ameise haben, hängt von
den Gewichtunksfaktoren $\alpha$ und $\beta$ ab. Wird $\alpha$ beispielsweise
auf 0 gesetzt, so fließen die Informationen des kollektiven Gedächtnisses und
damit die Pheromoninformationen nicht in die Wegfindung mit ein. Eine Ameise
wird also einen Weg wählen, der nur auf der aktuellen Distanz zu einem nächsten
Knotenpunkt basiert. Damit gleicht das Suchen eines Optimums einer Suche mit
Bruteforce und ist nicht effizient. Wird hingegen $\beta$ auf 0 gesetzt, so
berücksichtigt eine Ameise nur die Pheromoninformationen, wodurch keine neuen
Wege mehr exploriert werden. Es besteht die Gefahr, dass das globale Optimum
nicht gefunden wird, da jegliches individuelle Verhalten fehlt.

\section{Parameterabhängigkeiten}

Für den ACO können diverse Startparameter definiert werden. Neben den
bereits ausführlich erläuterten Parametern $\alpha$ und $\beta$, kann
mithilfe des Parameters \emph{evaporation} Angegeben werden, wie viel Prozent
des ursprünglichen Pheromonwerts nach einer Iteration behalten werden. Damit
wird das Verfliegen des Duftstoffes simuliert. Ein höherer Wert lässt die
Pheromonenwerte kleiner werden, dadurch kann angepasst werden, wie stark
auch niedrigfrequentierte Pfade in der nächsten Iteration zur Suche nach einem
Optimum miteinbezogen werden. Es sollten eher geringe Werte gewählt werden.
Bei einer \emph{evaporation} von 1.0 (100\%) werden nach jeder Iteration die
Pheromone zurückgesetzt, wodurch das kollektive Gedächtnis nach jeder Iteration
gelöscht wird. Das hat zur Folge, dass der Algorithmus nicht das globale
Optimum finden kann und Ameisen eine Bruteforce-Suche im gesamten Suchraum
starten.

Mit einem weiteren Parameter $q$ kann definiert werden, wie viele
Pheromone von einer einzelnen Ameise auf einem Pfad liegengelassen wird. Ein
höherer Wert führt zu einer höheren \glqq Auflösung\grqq in der Pheromonmatrix.
Wird beispielsweise $q=1$ gewählt und es laufen 10 Ameisen eine Kante entlang,
beträgt der Pheromonwert dieser Kante nach der ersten Iteration zunächst 10.
Durch die \emph{evaporation}, die in diesem Beispiel auf 0.1 (10\%) gesetzt
werden soll, wird der Pheromonwert dieser Kante schließlich den Wert 9
annehmen. Die Differenz zwischen diesen beiden Werten ist 1.
Wird hingegen $q=10$ gewählt, so nimmt der Pheromonwert dieser Kante den
Wert 100, bzw. nach der Verdampfung 90 an. Die Differenz der beiden Werte
(vor und nach der Verdampfung) beträgt hier bereits 10.
Das hat zur Folge, dass höhere Werte für $q$ eine größere Unterscheidung von
Pheromonwerten in der Matrix hat, wodurch die Intensität, die ein potentiell
guter Pfad für das Optimum auf das kollektive Gedächtnis hat, verstärkt wird.

Mit dem \emph{antFactor} lässt sich berechnen, wie viele Ameisen in Abhänigkeit
von Knotenpunkten für den Algorithmus erzeugt werden. Hier sollte ein Wert
zwischen 0.6 und 0.8 gewählt werden. Ein Wert von 0.6 würde bedeuten, dass für
60\% von Knotenpunkten, Ameisen erstellt werden, bei einer Anzahl von 100
Knoten in einem Graphen also 60 Ameisen zur Verfügung stehen, um das Optimum
für diesen Graphen zu finden. Mit diesem Parameter kann die Effizient des
Algorithmus gesteigert werden, da nicht für jeden Knotenpunkt eine Ameise zur
Verfügung stehen muss, wodurch die Komplexität reduziert und die
Geschwindigkeit bei der Ausfürhung des ACO erhöht wird.

Der \emph{randomFactor} kann verwendet werden, um bei der Auswahl von
nächsten Knotenpunkten einen kleinen Zufall einzubauen, wodurch die Exploration
erhöht wird, da Ameisen nicht unbedingt der nach der in Abschnitt
\ref{sec:prozedur} vorgestellten Wahrscheinlichkeitsberechnung besten Weg
nehmen müssen.

Zuletzt kann mit dem Parameter \emph{maximumIterations} festegelgt werden,
wie viele Iterationen durchlaufen werden sollen, bevor der Algorithmus
terminiert wird. Dies verhindert, dass ein Algorithmus nicht terminiert, wenn
keine Lösung gefunden wird, bzw. die gefundene Lösung noch nicht dem globalen
Optimum entspricht, eine weitere Suche das Ergebnis aber nicht erheblich
verbessern würde.

\section{Zusammenspiel \emph{Ant} und \emph{AntColony}}

Bei einer Konfiguration von $\alpha = 2$ und $\beta = 2$, werden $\tau$ und
$\eta$ gleichermaßen berücksichtigt. Damit wird sowohl das kollektive
Gedächtnis (Pheromonenmatrix), als auch das individuelle Verhalten
(Metaheuristik) in die Wegfindung mit einbezogen. Dies führt zu einer
ausgewogenen Exploration und Exploitation des Suchraums. Mit einer
Konfiguration, in der die \emph{evaporation} auf 0.05 gesetzt wird, werden
die Werte in der Pheromonenmatrix nach jeder Iteration um 5\% gesenkt. Es
werden also nur 95\% der eigentlichen Pheromonenwerte für die nächste
Iteration übernommen. Das hat zur Folge, dass Wege, die einmal als
vielversprechend angesehen wurden, mit der Zeit auch wieder vergessen werden
können, wenn ein anderer, besserer Weg gefunden wird. Das erhöht die
Leistungsfähigkeit des ACO. Mit dem Wert \emph{q} = 500 wird für jede Ameise,
die eine Kante des Graphen entlanggeht, der Pheromonwert für diese Kante um
500 erhöht. Damit lässt sich eine gute Auflösung der Pheromonenmatrix
erzielen. Der \emph{antFactor} von 0.8 sorgt dafür, dass für 80\% des
Suchraums, Ameisen erzeugt werden, die den gesamten Suchraum erkunden. Im
Falle des TSP würden bei 100 Städten also die Ameisenkolonie eine Größe von
80 Ameisen annehmen, wobei
jede Ameise in jeder Iteration einen Pfad für das TSP sucht. Ein höherer Wert
erhöht die Komplexität pro Iteration, ein niedrigerer Wert erhöht hingegen
die Anzahl an Iterationen, dafür wird die Rechenaufwand pro Iteration
minimiert.
Ein \emph{randomFactor} von 0.01 fügt der Auswahl bei der Wegfindung von
Ameisen für den nächsten Knotenpunkt eine Variation hinzu. Mit einer
Wahrscheinlichkeit von 1\% wählt die Ameise also einen anderen Knoten, als
den, den sie aufgrund der Berechnung gewählt hätte.